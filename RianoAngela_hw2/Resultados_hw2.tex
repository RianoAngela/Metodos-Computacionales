\documentclass[a4paper]{article}
\usepackage{graphicx}
\usepackage{tabularx}
\usepackage{float}
\usepackage{enumerate}

\title{Resultados taller 2: Métodos Computacionales}

\author{Angela Johanna Riaño Sánchez}

\date{18 de julio de 2019}

\begin{document}
\maketitle

\section{Transformada de Fourier: Imágenes híbridas.}
En este ejercicio se presentan dos fotografias de un rostro feliz y otro serio, a continuación se veran las graficas que se realizaron a medida que se iban pasando por los filtros de fourier de dos dimensiones, así mismo como la imagen final 

\subsection{Gráficas}

\begin{figure}[H]
\includegraphics[scale=0.7]{cara2Original.png}
\caption{Cara 2 original}
\centering
\end{figure}

\begin{figure}[H]
\includegraphics[scale=0.7]{cara3Original.png}
\caption{Cara 3 original}
\centering
\end{figure}

\begin{figure}[H]
\includegraphics[scale=0.7]{nuevacara2.png}
\caption{Cara 2 al pasar por todos los filtros}
\centering
\end{figure}

\begin{figure}[H]
\includegraphics[scale=0.7]{nuevacara3.png}
\caption{Cara 3 al pasar por todos los filtros}
\centering
\end{figure}

Esta fotografia se puede explicar debido a que al superponer dos imagenes que han pasado por diferentes filtros de Fourier, es decir, la cara 2 se le aumento la resolución para que conservara mejor sus detalles al ver la foto de cerca. Sin embargo, la cara 3 se le aplico un filtro donde la resolucion baja de tal manera que parece borrosa y solo conserva ciertos detalles que el ojo humano podrá apreciar al tener una cierta distancia de la fotografia. Así se crea el efecto de tener 2 imagenes diferentes en una sola.

\begin{figure}[H]
\includegraphics[scale=0.7]{hibrida.png}
\caption{Fotografia hibrida}
\centering
\end{figure}


\subsection{Comparación de los distintos métodos de solución de ecuaciones diferenciales ordinarias: caso de una masa orbitando alrededor de otra}
En este ejercicio se compararon 3 metodos para poder resolver ecuaciones diferenciales ordinarias de segundo orden, tales como: Euler, Leap Frog y Runge Kutta de cuarto orden 

\begin{figure}[H]
\includegraphics[scale=0.7]{nuevacara3.png}
\caption{Cara 3 al pasar por todos los filtros}
\centering
\end{figure}

\begin{figure}[H]
\includegraphics[scale=0.7]{nuevacara3.png}
\caption{Cara 3 al pasar por todos los filtros}
\centering
\end{figure}



\end{document}
